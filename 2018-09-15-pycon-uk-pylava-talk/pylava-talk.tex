\documentclass{beamer}


% PACKAGES 
% ========
% Do not indent paragraphs and insert blank space between paragraphs.
\usepackage{parskip}

% Commands like \textasciigrave, \textquotesingle, etc. are required by
% listings when \lstset{upquote=true} is used. They are defined in
% textcomp.sty.
\usepackage{textcomp}

% To enter syntax-highlighted code blocks.
\usepackage{listings}


% THEME
% =====
% Set main theme.
\usetheme{CambridgeUS}

% Set outer color theme.
\usecolortheme{dolphin}

% Set inner color theme for blocks.
\usecolortheme{orchid}

% Leave more margin from left and right side.
\setbeamersize{
    text margin left=1.5em,
    text margin right=1.5em,
}

% Align the frame title with the text with additional margin.
\setbeamertemplate{frametitle}[default][leftskip=0.5em]

% Remove navigation symbols from footer.
\setbeamertemplate{navigation symbols}{}

% CambridgeUS uses infolines.sty for outer theme which in turn defines
% footline with three color boxes for author, title, and date. We set
% the footline to omit the color box for title and distribute the width
% of the remaining two color boxes equally across the paper width. The
% code below is a derivative of the \defbeamertemplate*{footline} code
% in infolines.sty.
\setbeamertemplate{footline}
{%
  \leavevmode%
  \hbox{%
  \begin{beamercolorbox}[wd=.5\paperwidth,ht=2.25ex,dp=1ex,center]
        {author in head/foot}%
    \usebeamerfont{author in head/foot}\insertshortauthor
                  {~~~~(\insertshortinstitute)}
  \end{beamercolorbox}%
  \begin{beamercolorbox}[wd=.5\paperwidth,ht=2.25ex,dp=1ex,right]
        {date in head/foot}%
    \usebeamerfont{date in head/foot}\insertshortdate{}\hspace*{2em}
    \usebeamertemplate{page number in head/foot}\hspace*{2ex} 
  \end{beamercolorbox}}%
  \vskip0pt%
}


% LINKS
% =====
% Color links blue.
\hypersetup{colorlinks=true,urlcolor=blue}

% Use current text font for links (override default teletype font).
\urlstyle{same}


% CODE LISTINGS
% =============
% Colors for syntax highlighting.
\definecolor{codecolor}{HTML}{000060} % dark blue
\definecolor{commentcolor}{HTML}{586e75} % bluish gray
\definecolor{keywordcolor}{HTML}{0000f0} % blue
\definecolor{stringcolor}{HTML}{600000} % dark red
\definecolor{placeholdercolor}{HTML}{006000} % dark green

% Base style for code listing.
\lstset{
    % Use small teletype font (override default roman font).
    basicstyle=\ttfamily\footnotesize\color{codecolor},
    % Use natural width of characters and do not mess with alignment.
    columns=fullflexible,
    % Do not drop consecutive spaces.
    keepspaces=true,
    % Use straight quotes instead of curved quotes.
    upquote=true,
    % Do not show spaces in strings as visible spaces.
    showstringspaces=false,
    % Set colors for syntax highlighting.
    commentstyle=\color{commentcolor},
    keywordstyle=\color{keywordcolor},
    stringstyle=\color{stringcolor},
}

% Plain code does not have any syntax highlighting.
\lstnewenvironment{plaincode}{}{}

% Display inline code on a gray background.
\newcommand{\inlinecode}[1]{
    \colorbox{black!10}{%
        \texttt{#1}%
    }
}

% Title page information.
\title{From Pylama to Pylava}
\subtitle{\vspace{2mm} Building a Community Maintained Fork of Pylama}
\author{Susam Pal}
\institute[PyCon UK 2018, Cardiff, UK]{
    PyCon UK 2018, Cardiff City Hall, Cardiff, UK
}
\date{15 Sep 2018}


\begin{document}

% Title
\frame{\titlepage}


\begin{frame}
    \frametitle{About Me}

    \begin{itemize}
        \item Security software developer working at Walmart Labs.
        \item Developing security microservices in Python and Go.
        \item Maintainer of Pylava on Fridays.
    \end{itemize}
\end{frame}


% What is Pylama?
\begin{frame}
    \frametitle{What is Pylama?}
    \begin{itemize}
        \item Pylama is a code audit tool for Python and JavaScript.
        \item Wraps around other popular linters and code quality tools.
        \item Uses pycodestyle, pydocstyle, PyFlakes, McCabe, etc.
        \item Supports Pylint and Gjslint via extension packages.
        \item Written and maintained by Kirill Klenov since 2012.
        \item Maintained until 2017.
    \end{itemize}
\end{frame}


% Why Fork Pylava?
\begin{frame}[fragile]
    \frametitle{Why Fork Pylama?}

    \begin{itemize}
        \item
        Need for maintenance updates.
        \item
        Python 3.6 warned that \inlinecode{async} would become a
        keyword.
        \item
        In Python 3.7, \inlinecode{async} did become a reserved keyword.
        \item
        Pylama broke!
    \end{itemize}
    
    \begin{block}{SyntaxError!}
\begin{plaincode}
$ pylama
Traceback (most recent call last):
  File "venv/bin/pylama", line 7, in <module>
    from pylama.main import shell
  File "venv/lib/python3.7/site-packages/pylama/main.py", line 10
    from .async import check_async
          ^
SyntaxError: invalid syntax
\end{plaincode}
    \end{block}
\end{frame}


\begin{frame}
    \frametitle{Forking Pylama to Pylava}

    \begin{itemize}
        \item
        Pull requests to fix issues were already available from
        contributors.

        \item
        Forked Pylama to Pylava.

        \item
        Pull requests sent to Pylama were merged into Pylava.

        \item
        Pylava is now maintained as a community project.

        \item
        Forked Pylama\_pylint to Pylava\_pylint.

        \item
        Pylava\_pylint is also maintained as a community project.

        \item
        Pylama\_gjslint is not available with Pylava yet.
    \end{itemize}
\end{frame}

\begin{frame}[fragile]
    \frametitle{Use Pylava}
    
    \begin{itemize}

    \item
    Pylava is now available via its own PyPI entry at
    \url{https://pypi.org/project/pylava/}.

    \item
    Install it with \inlinecode{pip} and enter the command
    \inlinecode{pylava} to run it.

\lstset{basicstyle=\ttfamily\normalsize\color{codecolor}}
\begin{plaincode}
$ pip install pylava
$ pylava
\end{plaincode}

    \item No output is good.

    \end{itemize}
\end{frame}


\lstset{basicstyle=\ttfamily\tiny\color{codecolor}}

\begin{frame}[fragile]
    \frametitle{Example Report}
\begin{plaincode}
~/git/pylava$ pylava
config.py:24:80: E501 line too long (89 > 79 characters) [pycodestyle]
config.py:123:80: E501 line too long (81 > 79 characters) [pycodestyle]
config.py:151:80: E501 line too long (88 > 79 characters) [pycodestyle]
core.py:14:1: C901 'run' is too complex (17) [mccabe]
core.py:92:5: E731 do not assign a lambda expression, use a def [pycodestyle]
core.py:95:9: E731 do not assign a lambda expression, use a def [pycodestyle]
main.py:13:1: C901 'check_path' is too complex (11) [mccabe]
main.py:28:80: E501 line too long (87 > 79 characters) [pycodestyle]
main.py:38:80: E501 line too long (84 > 79 characters) [pycodestyle]
lint/pylava_radon.py:27:80: E501 line too long (90 > 79 characters) [pycodestyle]
lint/pylava_radon.py:28:80: E501 line too long (83 > 79 characters) [pycodestyle]
lint/pylava_radon.py:31:1: W391 blank line at end of file [pycodestyle]
lint/pylava_pyflakes.py:26:1: E303 too many blank lines (3) [pycodestyle]
lint/pylava_mccabe.py:22:80: E501 line too long (89 > 79 characters) [pycodestyle]
lint/extensions.py:38:1: E402 module level import not at top of file [pycodestyle]
lint/pylava_pydocstyle.py:12:1: E402 module level import not at top of file [pycodestyle]
lint/pylava_pydocstyle.py:29:80: E501 line too long (90 > 79 characters) [pycodestyle]
lint/pylava_pycodestyle.py:2:80: E501 line too long (80 > 79 characters) [pycodestyle]
libs/inirama.py:29:1: C901 'TryExcept 29' is too complex (12) [mccabe]
libs/importlib.py:6:1: E302 expected 2 blank lines, found 1 [pycodestyle]
libs/importlib.py:11:1: E0602 undefined name 'xrange' [pyflakes]
libs/importlib.py:16:31: E127 continuation line over-indented for visual indent [pycodestyle]
\end{plaincode}
\end{frame}


\begin{frame}
    \frametitle{Project URLs}

    GitHub: \url{https://github.com/pylava/pylava}

    Documentation: \url{https://pylavadocs.readthedocs.io/}

    PyPI: \url{https://pypi.org/project/pylava/}
\end{frame}


\begin{frame}
\Huge
\centering

Thank You!
\end{frame}

\end{document}
